\documentclass[12pt, letterpaper]{article}

\usepackage[utf8]{inputenc}
\usepackage{mathtools}
\usepackage[a4paper, total={6in, 9in}]{geometry}

\title{CSE 21 HW 3}
\author{Brian Masse}
\date{January 29, 2025}

\begin{document}

\maketitle
\newpage

% MARK: Problem 1
% problem statement
\bf{ 1. Consider the set H of all non-negative integer solutions to the equation: }

\[ a_{1} + a_{2} + a_{3} + a_{4} + a_{5} + a_{6} = 30  \]

\-\ \newline
\textnormal{You wish to pick one of the elements of H at random and the way you decide to do it is the following:
Initialize a 6-tuple to be all zeros (0,0,0,0,0,0) Roll a fair 6 sided die 30 times. For each roll that shows a 1, increment the 1st position, for each roll
that shows a 2, increment the 2nd position and so on}

% MARK: Problem 1a
\-\ \newline
\-\ \newline
\-\ \it{ a) Justify that it is possible for any element of H to be produced using this sampling process }

\-\ \newline
\textnormal{There are \( {30 + 6 - 1 \choose 6 - 1} = {35 \choose 5} \) distributions. (As concluded using Stars and Bars method)}

\-\ \newline
\textnormal{ There are \(6^{30}\) possibilities using the dice rolling sampling. }

\-\ \newline
\textnormal{ Because \({35 \choose 5} < 6^{30}\) it is possible to generate every possible distribution using this sampling process. (It is possible to construct a function that maps a subset of the dice results to each event in the sample space.)}

% MARK: Problem 1b
\-\ \newline
\-\ \newline
\-\ \it{ b) Using this sampling process, what is the probability that the first entry is 0 or the last entry is 0? }

\[ 2 \cdot \frac{5^{30}}{6^{30}} \approx 0.0084\]

\begin{itemize}
    \item \(2\) : There are two possibilities: either roll no 6s, or no 1s.
    \item \({5^{30}}\) : Ways to roll only 5 numbers 30 times.
    \item \(6^{30}\) : Total ways to roll all 6 dice 30 times.
\end{itemize} 

% MARK: Problem 1c
\-\ \newline
\-\ \newline
\-\ \it{ c) What is the probability that the first entry is 15 or the last entry is 15? }

\[ 2 \cdot \frac{{30 \choose 15} \cdot 5^{15}}{6^{30}} \approx 4.282 \cdot 10^{-5}\] 

\begin{itemize}
    \item \(2\) : There are two possibilities: Either roll 15 1s or 15 6s
    \item \({{30 \choose 15}}\) : Ways to pick out 15 of the 30 dice rolls to be either 1 or 6.
    \item \(5^{15}\) : Ways to roll only 5 number 15 times.
    \item \(6^{30}\) : Total ways to roll all 6 dice 30 times.
\end{itemize} 

% MARK: Problem 1d
\-\ \newline
\-\ \newline
\-\ \it{ d) What is the probability that at least one entry is greater than 15? }

\[ 6 \cdot \frac{ \sum_{k=15}^{30} {30 \choose k} \cdot 5^{30 - k} }{6^{30}} \approx 0.000157\]

\begin{itemize}
    \item \(6\) : There are 6 ways to pick one number to roll at least 15 times
    \item \({\sum_{k=15}^{30} {30 \choose k} \cdot 5^{30 - k}}\) : Ways to roll at least 15 of a certain number
    \item \(6^{30}\) : Total ways to roll all 6 dice 30 times.
\end{itemize} 

% MARK: problem 1e
\-\ \newline
\-\ \newline
\-\ \it{ e) Explain why this is not a uniform random sampling.  }

\-\ \newline
\-\ \newline
\textnormal{Consider the probability of \(a_1\) being 15.}

\-\ \newline
\textnormal{a) for a uniform distribution:}

\[ p_0(a_0 = 15) = \frac{{30 - 15 + 5 - 1 \choose 5 - 1}}{{31 \choose 5}} = \frac{{19 \choose 4}}{{35 \choose 5}} \approx 0.012\]

\-\ \newline
\textnormal{b) for the sampling procedure:}

\[ p_1(a_0 = 15) = \frac{{30 \choose 15} \cdot 5^{15}}{6^{30}} \approx 2.141\cdot10^{-5}\]

\-\ \newline
\-\ \newline{\(p_0 \neq p_1\), thus this cannot be a uniform sampling}

% MARK: problem 1f
\-\ \newline
\-\ \newline
\-\ \it{ f) Which outcome or outcomes are the least likely to occur and why?  }

\-\ \newline
\textnormal{the outcomes where some \(a_i = 30\) are all equally the least likely to occur, since they can only be achieved by rolling 30 dice the same; the least likely event to occur.}


% MARK: Problem 2
% problem statement
\newpage
\bf{ 2. There are 3 bags, each with 10 colored balls. }

% MARK: Problem 2a
\-\ \newline
\-\ \newline
\-\ \it{ a) Suppose your friend gives you a bag uniformly at random. You reach in (with both hands) and
pick two balls (one in each hand) of the same color. What is the probability that you are holding
bag number 3?  }

\-\ \newline
\[p(A | B) = \frac{p(B | A)\cdotp(A)}{p(B)} \approx 0.256\]

\-\ \newline
\-\ \newline
A : Set of all bags that are bag 3, 
B : Set of all balls that are same color

\begin{itemize}
    \item \( p(A | B) \) : Probability of picking same colored balls from bag 3 = \(\frac{4}{9}\)
    \item \( p(A) \) : Probability of picking bag 3 = \(\frac{1}{3} \)
    \item \( p(B)\) : Probability of picking same colored Balls =
\end{itemize} 

\[ \frac{1}{3}( \frac{4}{9} + (\left(\frac{1}{5}\right)\left(\frac{1}{9}\right) + \left(\frac{4}{5}\right)\left(\frac{7}{9}\right)) + (\left(\frac{4}{5}\right)\left(\frac{7}{9}\right) + \left(\frac{1}{5}\right)\left(\frac{1}{9}\right)) ) \]

\[ = \frac{78}{45 * 3}\]


% MARK: Problem 2a
\-\ \newline
\-\ \newline
\-\ \it{ b) Suppose your friend gives you a bag uniformly at random. You reach in (with both hands) and
pick two balls (one in each hand) of different colors. What is the probability that you are holding
bag number 3?  }

\-\ \newline
\[p(A | C) = \frac{p(C | A)\cdotp(A)}{p(C)} \approx 0.4386\]


\-\ \newline
\-\ \newline
A : Set of all bags that are bag 3, 
C : Set of all balls that are different color

\begin{itemize}
    \item \( p(A | C) \) : Probability of picking different colored balls from bag 3 = \(\frac{5}{9}\)
    \item \( p(A) \) : Probability of picking bag 3 = \(\frac{1}{3} \)
    \item \( p(B)\) : Probability of picking different colored Balls =
\end{itemize} 

\[ \frac{1}{3}( \frac{5}{9} + 2 \cdot (\left(\frac{1}{5}\right)\left(\frac{8}{9}\right) + \left(\frac{4}{5}\right)\left(\frac{2}{9}\right))) \]
\[ = \frac{57}{45 * 3}\]


% MARK: Problem 3a
% problem statement
\newpage
\bf{ 3. Consider the set defined for any positive integer \( k: N_k = \{1,2,3,..k-1, k \} \)  }

\-\ \newline
\-\ \newline
\-\ \it{ a) Suppose I pick a 3 element subset from \(N_9\) uniformly at random. What is the probability that
the median element in my subset is in the set \(\{4, 5, 6\}\)  }

\[ p = \frac{46}{{9 \choose 3}} = \frac{46}{84} \]

\begin{itemize}
    \item number of subsets with median 4: \(3\cdot5\)
    \item number of subsets with median 5: \(4\cdot4\)
    \item number of subsets with median 6: \(5\cdot3\)
\end{itemize} 

\-\ \newline
\begin{itemize}
    \item Sets with 4, 5, 6 as the median: 15 + 15 + 16 = 46
    \item Total number of length 3 sets: \({9 \choose 3}\)
\end{itemize} 

% MARK: Problem 3b
\-\ \newline
\-\ \it{ b) Suppose I pick a sequence of length 3 such that each entry of the sequence is an element from \(N_9\)
independently chosen uniformly at random. What is the probability that the median element in
my sequence is in the set \(\{4, 5, 6\}\)  }

\[ p = \frac{243}{9^{3}} = \frac{243}{729} \]

\begin{itemize}
    \item number of subsets with median 4: \(9\cdot9\)
    \item number of subsets with median 5: \(9\cdot9\)
    \item number of subsets with median 6: \(9\cdot9\)
\end{itemize} 

\-\ \newline
\begin{itemize}
    \item Sequences with 4, 5, 6 as the median: \(3 \cdot 9^{2} = 243\)
    \item Total number of length 3 sequences: \(9 ^{3}\)
\end{itemize} 


% MARK: Problem 3c
\-\ \newline
\-\ \it{ c) For an arbitrary positive integer n, suppose I pick a 3 element subset from \(N_3n\) uniformly at
random. Derive a formula for the probability that the median element in my subset is in the
middle third of the set, i.e \(\{n + 1, n + 2, ..., 2n\}\)  }

\-\ \newline
\textnormal{ Number of subsets with some median \(k \in \{n + 1, n + 2, ..., 2n\}\) }
\[ (k - 1)(3n - k) \]

\-\ \newline
\textnormal{All subsets with median belonging to \(\{n + 1, n + 2, ..., 2n\}\)}
\[ \sum_{k=n+1}^{2n} (k - 1)(3n - k)\]

\-\ \newline
\textnormal{All possible sets}
\[ {3n \choose 3} \]

\-\ \newline
\textnormal{thus,}
\[ p(n) = \frac{ \sum_{k=n+1}^{2n} (k - 1)(3n - k) }{ {3n \choose 3 } } \]


% MARK: Problem 3d
\-\ \newline
\-\ \it{ d) For an arbitrary positive integer n, suppose I pick a sequence of length 3 such that each entry
of the sequence is an element from \(N_3n\) independently chosen uniformly at random. What is
the probability that the median element in my sequence is in the middle third of the set? }

\-\ \newline
\textnormal{ Number of sequences with some median \(k \in \{n + 1, n + 2, ..., 2n\}\) }
\[ (3n)^{2} \]

\-\ \newline
\textnormal{ All sequences with median belonging to \(\{n + 1, n + 2, ..., 2n\}\) }
\[ {\sum_{k=n+1}^{2n} (3n)^{2} } \]

\-\ \newline
\textnormal{ Total number of sequences }
\[ {(3n)^{3}} \]

\-\ \newline
\textnormal{ Thus, }
\[ p(n) = \frac{ \sum_{k=n+1}^{2n} (3n)^{2} }{ (3n)^3 } = \frac{9\cdot n^{3}}{27 \cdot n^{3}} = \frac{1}{3}\]



% MARK: Problem 4a
% problem statement
\newpage
\bf{ 4. }

\-\ \it{ a) Consider the trial of rolling two fair 6-sided dice and the random variable X is the absolute value
of the difference of the two dice rolls. Compute the expected value of X  }

\begin{eqnarray}
E(X) &=& \sum_{s \in S}^{} p(s)X(s) \\
&=& \sum_{r \in X(S)}^{} r\cdot p(X=r) \\
&=& \sum_{k=0}^{5} k\cdot \frac{2\cdot(6-k)}{36} \\
&\approx& 1.944
\end{eqnarray}

\begin{itemize}
    \item k = 0 to 5 : There are only 6 possible outcomes for the absolute value of the difference of dice rolls: \{0, 1,2,3,4,5\}
    \item r = k
    \item p(X=r) = \( \frac{2\cdot(6-k)}{36} \), for each outcome r \(\in\) X(S), there are \(2\cdot (6-k)\) dice roll combinations, out of the 36 total possible combinations
\end{itemize} 

% MARK: Problem 4b
\-\ \newline
\-\ \it{ b) Suppose you are at a casino and you are playing roulette and your strategy is to bet 1 dollar on
red You have a \(\frac{18}{38}\) probability of winning 1 dollar and \(\frac{20}{38}\) probability of losing 1 dollar.
Suppose you start with 1 dollar and you play until you have no money left. What is the expected
number of games you get to play? }

\begin{eqnarray}
    E(X) &=& E(X | B)P(B) + E(X | \overline{B})P(\overline{B}) \\
    &=& 1 \cdot \left( \frac{18}{38} \right) - 1 \cdot \left( \frac{20}{38} \right) \\
    &=& -\frac{1}{19}
\end{eqnarray}

\textnormal{Thus, you should expect to play 19 games before losing your 1 initial dollar. }

\begin{itemize}
    \item X : Expected amount of money earned after each game
    \item B : Winning a game of Roulette
    \item \(\overline{B}\) : Losing a game of Roulette
    \item \(E(X | B)\) Expected money given that you win = 1
    \item \(E(X | \overline{B})\) Expected money given that you lose = -1
\end{itemize} 

\end{document}
