\documentclass[12pt, letterpaper]{article}

\usepackage[utf8]{inputenc}
\usepackage{mathtools}
\usepackage[a4paper, total={6in, 9in}]{geometry}

\title{CSE 21 Hw 1}
\author{Brian Masse}
\date{January 13, 2025}

\begin{document}

\maketitle
\newpage

% MARK: Problem 1a
% problem statement
\bf{ 1. (a) Use regular induction to show that for all n $\ge$ 1 that }
\[ \sum_{k=1}^{n} \frac{1}{(k)(k+1)} = \frac{n}{n+1}  \]

% proving the base case
\-\ \it{ i) Prove the base case: (n = 1) }

\[ \sum_{k=1}^{1} \frac{1}{k( k + 1 )} = \frac{1}{2} = \frac{1}{1 + 1} \]

% assuming the general case
\-\ \it{ ii) Assume for some n = j  }
\[ \sum_{k=1}^{j} \frac{1}{k( k + 1 )} = \frac{j}{j + 1}\]

% showing the induction
\-\ \it{ and show that it holds for j + 1  }

\begin{eqnarray}
&& \sum_{k=1}^{j + 1} \frac{1}{k( k + 1 )}\\
&=& \frac{1}{(j + 1)(j + 2)} + \frac{j}{j + 1}\\
&=& \frac{1 + j(j + 2)}{(j + 1)(j + 2)} \\
&=& \frac{ (j + 1)^{2} }{(j + 1)(j + 2)} \\
&=& \frac{ (j + 1) }{(j + 2)}
\end{eqnarray}

% MARK: Problem 1b
% problems state
\-\ \newline
\bf{ (b) Use regular induction to show that for all n $\ge$ 0 that }
\[ \sum_{k=0}^{n} 2^{k} < 2^{n + 1}  \]

% proving the base case
\-\ \it{ i) Prove the base case: (n = 0) }

\[ \sum_{k=0}{n} 2^{k} = 1 < 2 \]


% prove the general case
\-\ \it{ ii) Assume for some n = j  }

\[ \sum_{k=0}{n} 2^{k} = 2^{n + 1}\]

\-\ \it{ and show that it holds for j + 1  }

\begin{eqnarray}
&& \sum_{k=0}^{k} 2^{j + 1} \\
&<& 2^{j + 1} + 2^{j + 1} \\
&=& 2*2^{j +1 } = 2^{j + 2}
\end{eqnarray}


% MARK: problem 2a
\newpage
\bf{ 2. A password is a string over the alphabet of the 26 uppercase letters, the 26 lowercase letters, and the 10 digits }

\it{ (a) How many 8-char passwords start and end with a digit?  }

\[ k = 10\cdot (26 + 26 + 10)^{6} \cdot 10 = 10^{2} \cdot 62^{6}\]

\begin{itemize}
    \item \(10\) : number of ways to start the string with a number
    \item \((26 + 26 + 10)^{6}\) number of 6-char strings
    \item \(10 \) : number of ways to end the string with  a number
\end{itemize} 

% MARK: problem 2b
\-\ \newline
\it{ (b) How many 8-character passwords have exactly one uppercase and exactly one lowercase letter?  ( and the rest are digits ) }

\[ k = 26^{2} \cdot {8 \choose 2} \cdot 10^{6} \]

\begin{itemize}
    \item \(26^{2}\) : Number of ways to choose an uppercase and lower case letter
    \item \( {8 \choose 2} \) : Number of ways to arrange an 2 elements in the string
    \item \(10^{6} \) : number of 6 char strings with only digits
\end{itemize}


% MARK: problem 2c
\-\ \newline
\it{ (c) How many 8-character passwords avoids having the word COUNT (with any combinationof upper and lowercase letters)  }

\[ k = (26 + 26 + 10)^{8} - 2^{5} \]

\begin{itemize}
    \item \((26 + 26 + 10)^{8}\) : Total number of length 8 strings
    \item \((2^{5}) \) : number of case permutations of the word COUNT
\end{itemize}

% MARK: problem 2d
\-\ \newline
\it{ (d) How many 8-character passwords consist of three different characters with 3 copies of two chracters and 2 copies of the other?  }

\[ k = {62 \choose 3} \cdot {3 \choose 2} \cdot \left( \dfrac{8!}{3!3!2!} \right) \]

\begin{itemize}
    \item \(62 \choose 3\) : Number of ways to choose 3 characters from the alphabet
    \item \(3 \choose 2\) : Number of ways to pick 2 characters from 3 characters
    \item \(\frac{8!}{3!3!2!}  \) : Number of anagrams of the 3 characters (given the configuration specified by the problem)
\end{itemize}


% MARK: problem 2e
\-\ \newline
\it{ (e) How many 8-character passwords consist of 8 different letters that are in alphabetical order such that each letter can be uppercase or lowercase? }
\[ k =  {26 \choose 8} \cdot 1 \cdot 2^{8}\]

\begin{itemize}
    \item \(26 \choose 8\) : Number of ways to choose pick 8 unique characters from the english alphabet
    \item \(1\) :Number of ways to arrange them in correct alphabetical order
    \item \(2^{8}\) : Number of combinations of upper and lowercase characters. 
\end{itemize}


% MARK: Problem 3a
\newpage
\bf{ 3. For each expression, describe a set of objects that is counted by the expression and include your reasoning }

\-\ \newline
\it{ (a) }
\[ 8 \cdot 26 \cdot (26 + 10)^{7} \]

\emph{A string, starting with a number 0 - 7, followed by an english character, followed by 7 characters, either number or letter}

\begin{itemize}
    \item \(8\) : States the string must start with a digit belong to \{ 0, 1, 2, 3, 4, 5, 6, 7 \}
    \item \(26\) : States the second element in the string must be a letter in the english alphabet
    \item \(36^{7}\) : Number of length 7 strings using numbers and letters
\end{itemize}

% MARK: Problem 3b
\-\ \newline
\it{ (b) }
\[ 10^{2}(26 + 26)^{6} + 26^2(10 + 26)^6 \]

\emph{the number of \\\\
a) strings of length 8 that start with 2 numbers, and the rest are upper/lowercase letters \\\\
b) strings of length 8 that start with 2 uppercase letters, and the rest are lowercase letters or numbers}

\begin{itemize}
    \item \(10^{2}\) : States that the string (a) must start with 2 numbers
    \item \((26 + 26)^{6}\) : States the remainder is a length 6 string with upper and lowercase letters.
    \item \(26^{2}\) : States that the string (b) must start with 2 uppercase letters
    \item \( (10+26)^{6} \) : States that the remainder is a length 6 string with lowercase letters and numbers
\end{itemize}


% MARK: Problem 3c
\-\ \newline
\it{ (c) }
\[ (26 + 26 + 10)^{8} - 10^{8} \]

\emph{Number of length 8 strings with at least one letter}

\begin{itemize}
    \item \((26 + 26 + 10)^{8}\) : Number of length 8 strings
    \item \(10^{8}\) : Number of length 8 strings without a letter
\end{itemize}

% MARK: Problem 3d
\-\ \newline
\it{ (d) }
\[ (4!)2^{4} \]

\emph{ Number of length 8 strings starting with an anagram of 4 unique characters, and ending with a length 4 binary string }

\begin{itemize}
    \item \(4!\) : Number of Anagrams of 4 unique characters
    \item \(2^{4}\) : Number of length 4 binary strings
\end{itemize}


% MARK: 4a
\newpage
\bf{ 4. (a) How many different ways are tehre to arrange the letters in DISCRETEMATHEMATICS }

\[ k = \frac{18!}{ 2!2!2!3!3!2!2! }  \]

\-\ \newline
\textnormal{I : 2, S : 2, C : 2, E : 3, T : 3, M : 2, A : 2}


% MARK: 4b
\-\ \newline
\bf{ (b) How many different ways are there to color the 12 verticies of a hexagonal prism with 12 different colors? } \newline

\textnormal{\it{ For each coloring, there are \\
  6 rotations along the y axis \\
+ 2 rotations along the x axis \\
+ 2 rotations along the z axis \\
= 10 total rearangements of one color configurations}} 

\-\ \newline
\textnormal{Number of unique colorings = \( \frac{12!}{24} \)}


% MARK: 4c
\-\ \newline
\bf{ (c) How many different ways are there to color the 6 edges of a tetrahedron with 6 colors? } \newline

\textnormal{\it{ For each coloring, there are \\
3 distinct rotations along the y-axis \\
3 distinct rotations along the z-axis
}}

\-\ \newline
\textnormal{ Number of unique colorings = \( \frac{6!}{12} \) }

\end{document}