\documentclass[12pt, letterpaper]{article}

\usepackage[utf8]{inputenc}
\usepackage{mathtools}
\usepackage[a4paper, total={6in, 9in}]{geometry}

\title{CSE 21 Hw 2}
\author{Brian Masse}
\date{January 22, 2025}

\begin{document}

\maketitle
\newpage

% MARK: Problem 1a
% problem statement
\bf{ 1. (a) Suppose you are a painter and you have 18 different paintings and there are 6 different galleries
that are interested in showing your paintings }

% problem I
\-\ \newline
\-\ \it{ i) How many ways can you distribute not necessarily all 18 paintings to the 6 galleries. }

\[ \sum_{k=0}^{18} {18 \choose k}6^{k} \]

\begin{itemize}
    \item \(\sum_{k=0}^{18}\) : Iterate through all the possible number of paintings to distribute
    \item \({18 \choose k}\) : Number of ways to select k paintings
    \item \(6^{k}\) : number of ways to distribute k paintings to 6 galleries
\end{itemize} 

% problem II
\-\ \newline
\-\ \it{ ii) How many ways can you distribute all 18 paintings to the 6 galleries so that each gallery gets
at least one painting?  }
\[ \sum_{k=0}^{6}(-1)^{6-k} {6 \choose k} k^{18}\]


\textnormal{The function mapping all 18 paintings to the 6 galleries must be onto, which requires the use of principle inclusion exclusion. }


% problem III
\-\ \newline
\-\ \it{ iii) How many ways can you distribute all 18 paintings to the 6 galleries if at least 1 gallery gets
exactly 9 paintings? } 
\[ {18 \choose 9} \cdot {6 \choose 1} \cdot 5^{9}  \]

\begin{itemize}
    \item \({18 \choose 9}\) : Choose 9 paintings for the one gallery
    \item \({6 \choose 1}\) : Pick the gallery to receive the 9 paintings
    \item \(5^{9}\) : Distribute the Remaining Paintings
\end{itemize} 

% MARK: Problem 1b
% mark: I
\bf{ 1. (b) Suppose you are a woodcut printmaker and you have 18 identical woodcut prints and there are 6
different galleries that are interested in showing your prints }

\-\ \newline
\-\ \it{ i) How many ways can you distribute not necessarily all 18 identical prints to the 6 galleries?  } 
\[ \sum_{k=0}^{18} {k+5 \choose 5}  \]

\begin{itemize}
    \item \(\sum_{k=0}^{18} \) : Iterate through all the possible number of prints to distribute
    \item \({k+5 \choose 5}\) : Number of ways to distribute k identical prints to 6 places
\end{itemize} 


% part II
\-\ \newline
\-\ \it{ ii) How many ways can you distribute all 18 identical prints to the 6 galleries so that each gallery
gets at least one print? } 
\[ {18 - 6 + 5 \choose 5} = {17 \choose 5}  \]

\begin{itemize}
    \item \({17 \choose 5} \) : Distribute 6 of the prints to each of the galleries, then distribute the remaining among the 6 galleries
\end{itemize} 


% part III
\-\ \newline
\-\ \it{ ii) How many ways can you distribute all 18 identical prints to the 6 galleries if at least 1 gallery
gets exactly 9 prints? }

\[ {6 \choose 1} \cdot {9 + 4 \choose 4} \]

\begin{itemize}
    \item \({6 \choose 1} \) : Number of ways to choose the gallery to receive 9 prints
    \item \({13 \choose 4} \) : Number of ways to distribute the remaining prints to the remaining galleries
\end{itemize} 



% MARK: Problem 2a
\newpage
\-\ \newline
\bf{ 2. Suppose you are traveling from the bottom-left corner of a 10 x 12 grid of blocks and you wish to get
to the top-right corner only using up and right movements }

\-\ \newline
\-\ \it{ a) How many paths are there from the bottom-left corner to the top-right corner using right and up
moves }

\[ {10 + 12 \choose 10} = \frac{22!}{10! \cdot 12!} \]

\begin{itemize}
    \item \(22 \) : Number of movement you must take to get from the start point to the end point.
    \item \(10\) : Number of different places you can choose to move up
\end{itemize} 

% Problem 2b
\-\ \newline
\-\ \it{ b) How many paths are there from the bottom-left corner to the top-right corner using right and up moves that
avoid passing through either blue dot? }

\[ {22 \choose 10} - (D_1 + D_2) + D_3 \]

\begin{eqnarray}
    D_1 &=& {3 + 4 \choose 3} \cdot {9 + 6 \choose 6} = {7 \choose 3}{15 \choose 6} \\
    D_2 &=& {7 + 7 \choose 7} \cdot {5 + 3 \choose 3} = {14 \choose 7}{8 \choose 3} \\
    D_3 &=& {7 \choose 3} \cdot {4 + 3 \choose 3} \cdot {8 \choose 3} = {7 \choose 3}{7 \choose 3}{8 \choose 3}
\end{eqnarray}

\begin{itemize}
    \item \( {22 \choose 10} \) : Total number of paths
    \item \( {D_1} \) : Number of paths going through at least \(P_1\)
    \item \( {D_2} \) : Number of paths going through at least \(P_2\)
    \item \( {D_3} \) : Number of paths going through both \(P_1\) and \(P_2\)
\end{itemize} 

\textnormal{ Subtracting the number of paths that at least go through \(P_1\) and the number of paths that at least go through \(P_1\) under counts the final value, since they both count the paths that go through both.  }
\textnormal{ Thus, you have to add back the number of paths that go through both }


% MARK: Problem 3a
\newpage
\-\ \newline
\bf{ 3. Compute the number of integer solutions for the equation }

\[ a_1 + a_2 + a_3 + a_4 = 26 \]

\-\ \newline
\-\ \it{ a) \( i \leq a_i  \) }

\[ {26 - 10 + 3 \choose 3} = {19 \choose 3}\]

\begin{itemize}
    \item \( {19 \choose 3} \) : 1 + 2 + 3 + 4 = 10, is the total amount that needs to be distributed to the various \(a_i\). The rest can be randomly distributed. 
\end{itemize} 


% MARK: Problem 3b
\-\ \newline
\-\ \it{ b)}
\( 0 \leq a_i\) for  \(a_i \in \{ 1, 2, 3, \}  \),
\( 5 \leq a_4 \leq 10\)

\[ {26 - 5 + 3 \choose 3} - {26 - 11 + 3 \choose 3 }\]
\[ = {24 \choose 3} - {18 \choose 3 }\]

\begin{itemize}
    \item \( {24 \choose 3} \) : Number of solutions for \(a_4 \geq 5\)
    \item \( {18 \choose 3 } \) : Number of solutions for \(a_4 > 10\)
\end{itemize} 

% MARK: Problem 3c
\-\ \newline
\-\ \it{ c) \( 0 \geq a_i \geq 7 \) }

\begin{eqnarray}
    &=& \sum_{k=0}^{4}(-1)^{4-k} {4 \choose k} {26 - 8k + 3 \choose 3} \\
    &=& \sum_{k=0}^{4}(-1)^{4-k} {4 \choose k} {29 - 8k \choose 3}
\end{eqnarray}


\textnormal{This solution uses the principle of inclusion/exclusion. It first counts all possible solutions \({29 \choose 3}\). Then for each \(a_i\), subtracts the solutions where at least \(a_i > 7\). Then it adds back solutions where for each pair \(a_i\), \(a_j\), both are greater than 7, and so on.}

\begin{itemize}
    \item \( \sum_{k=0}^{4}(-1)^{4-k} \) : Iterates through the number of \(a_i\) terms that are \(>\) 7. (For example, picking two \(a_i, a_j\), and counting all the solutions where at least both are \(>\) 7)
    \item \( {(-1)^{4-k} } \) : Oscillates between adding and subtracting the number of solutions per the principle of inclusion / exclusion
    \item \( {4 \choose k} \) : Number of ways to pick k \(a's\) out of the 4 total
    \item \( {29 - 8k \choose 3} \) : Uses 8k for the a's that are \(>\) 7, then distributes the remaining amount (26 - 8k) among the 4 a's
\end{itemize} 

% MARK: Problem 4a
\newpage
\-\ \newline
\bf{ 4. }

\-\ \newline
\-\ \it{ a) For integers n \(\geq\) k \(\geq\) 2, consider the identity: }  

\[ {n \choose k}k = {n - 1 \choose k - 1} n \]

\begin{itemize}
    \item \( LHS \) : Counts the number length n strings using three chars: \(\{ a_1, a_2, a_3 \}\). In this string, \(a_1\) is used exactly once, \(a_2\) appears exactly k - 1 times, and the rest is \(a_3\)
    \-\ \newline
    PROCEDURE : Create a length n string of all \(a_3\). Choose k out of the n spots to be \(a_2\), then out of those k spots, pick one to replace with \(a_3\)

    \item \(RHS\) : Counts the same set.
    \-\ \newline
    PROCEDURE : Create a length n string of all \(a_3\). Choose k-1 out of the first n-1 spots to be \(a_2\). Then out of the n spots, choose a character. If it is \(a_1\), replace it with \(a_3\). If it is \(a_2\), replace it with \(a_3\) and make the nth spot \(a_2\)

\end{itemize} 

% MARK: Problem 4b
\-\ \newline
\-\ \it{ b) For integers n \(\geq\) k \(\geq\) 0, consider the identity: }  

\[ {n + k - 1 \choose k - 1} = \sum_{j=0}^{k-1} {k \choose j} {n - 1 \choose k - 1 - j} \]

\begin{itemize}
    \item \( LHS \) : Counts the number of ways to distribute n indistinguishable objects among k unique groups

    \item \(RHS\) : Counts the same set by counting all the different distributions of containers with 0 objects.
    \-\ \newline
    \-\ \newline
    j = 0: 
    \-\ \newline
    - \( k \choose 0 \): number of ways to pick 0 containers to have 0 objects.
    \-\ \newline
    - \( { n - 1 \choose k -1 } = { n + k - 1 - k \choose k - 1} \): number of ways to distribute n objects to k containers such that each gets at least one
    
    \-\ \newline
    j = 1: 
    \-\ \newline
    - \( k \choose 1 \): Number of ways to pick 1 container to have 0 objects
    \-\ \newline
    - \( { n - 1 \choose k - 2 } = { n + (k - 1 - j) - (k - j) \choose k - 2} \): Number of ways to distribute n objects to k - 1 containers such that each gets at least one.

    \-\ \newline
    continues for all j \( \leq \) k - 1

\end{itemize} 


\end{document}